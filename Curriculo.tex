\documentclass[12pt, a4paper]{article}
\usepackage{times}			% Usa a fonte Latin Modern	
\usepackage[brazilian]{babel}		%data em português (eu acho)
\usepackage[T1]{fontenc}		% Selecao de codigos de fonte.
\usepackage[utf8]{inputenc}		% Codificacao do documento (conversão automática dos acentos)
\usepackage[top=0.5cm, bottom=2cm, left=3cm, right=1cm, margin=2cm]{geometry} %margens
\usepackage{multicol}   % para criar mais de uma coluna no texto
\usepackage{datetime}  %acjo que para usar data atual
\usepackage{hyperref}  %acjo que para usar data atual
\newcommand{\LlinhaG}{2pt} %definindo um comando ou variavel que é o caso
\newcommand{\TlinhaG}{17cm}  %definindo um comando ou variavel que é o caso
\newcommand{\LlinhaM}{1pt} 	 %definindo um comando ou variavel que é o caso
\newcommand{\TlinhaM}{17cm}		 %definindo um comando ou variavel que é o caso
\DeclareUnicodeCharacter{1EBD}{\~e}
\begin{document} % inicio do documento

\begin{center}
	\fontsize{14}{14}
	\textbf{LUIZ BERILO CASEMIRO QUEIROZ}
\end{center}

\begin{multicols}{2}
	\begin{flushleft}
		Rua Janaúba, nº 114,\\
		Bairro Mondubim - Fortaleza-CE\\
		Tel.: (85) 9-989071945\\
		E-mail: berilo.queiroz@gmail.com
		\url{https://github.com/beriloqueiroz}
		\url{https://www.linkedin.com/in/beriloqueiroz/}
	\end{flushleft}
	\vfill
	\begin{flushright}
		Data de nascimento: 03/04/1990\\
		Brasileiro\\
		Casado\\
	\end{flushright}
\end{multicols}

\begin{center} %linha grossa
	\rule{\TlinhaG}{\LlinhaG}
\end{center}

\begin{center}
	\textbf{OBJETIVO}\\
	Desenvolver atividades na área de Tecnologia da informação e computação, bem como atuar como desenvolvedor/programador FullStack, Backend, Frontend.
\end{center}

\begin{center}	%linha grossa
	\rule{\TlinhaG}{\LlinhaG}
\end{center}


\begin{center}
	\textbf{FORMAÇÃO ACADÊMICA}\\
\end{center}
\begin{itemize}
	\item[\textbf{2020 - 2021}] MBA em Gestão de Projetos e Processos\\
		INSTITUTO DE PÓS GRADUAÇÃO - IPOG
	\item[\textbf{2008 - 2015}] Graduação em ENGENHARIA DE MECATRÔNICA\\
		INSTITUTO FEDERAL DE CIÊNCIA E TECNOLOGIA DO CEARÁ, IFCE, Brasil\\
\end{itemize}

\begin{center}	%Linha não tão grossa - média
	\rule{\TlinhaG}{\LlinhaG}
\end{center}

\begin{center}
	\textbf{FORMAÇÃO COMPLEMENTAR}\\
\end{center}
\begin{itemize}
	\item[\textbf{2023 - 2023}] Curo de arquitetura Hexagonal – FullCycle\\
	\item[\textbf{2023 - 2023}] Segurança em aplicações WEB – Udemy\\
	\item[\textbf{2022 - 2022}] Módulo de Event Storming – FullCycle\\
	\item[\textbf{2022 - 2022}] Módulo de Cimunicação entre sistemas – FullCycle\\
	\item[\textbf{2022 - 2022}] Módulo de DDD – FullCycle\\
	\item[\textbf{2022 - 2022}] Módulo de SOLID Express – FullCycle\\
	\item[\textbf{2022 - 2022}] Módulo de Fundamentos de arquitetura de software – FullCycle\\
	\item[\textbf{2022 - 2022}] Módulo de Docker – FullCycle\\
	\item[\textbf{2021 - 2022}] HTML, CSS, Javascript – Rocketseat\\
	\item[\textbf{2021 - 2022}] nodejs, typescript, API REST (express) – Rocketseat - ignite\\
	\item[\textbf{2021 - 2022}] React Rocketseat - ignite\\
	\item[\textbf{2020 - 2021}] HTML – formação básica – linkedin learning\\
		Gestão de orçamentos em projetos – linkedin learning\\
		Gerenciamento de Cronogramas em projetos – linkedin learning\\
		Industria 4.0 Fundamentos da quarta revolução industrial – linkedin learning\\
		Fundamentos da Inteligência artificial: Aprendizado de Máquina. – linkedin learning\\
		Desenvolvimento de software remoto – linkedin learning
	\item[\textbf{2020 - 2021}] III Programa de formação de consultores – DOMANI\\
	\item[\textbf{2015 - 2015}] Elipse E3 Desenvolvedores. (Carga horária: 20h)\\
		Elipse Software, ELIPSE, Porto Alegre, Brasil
	\item[\textbf{2015 - 2015}] Curso de curta duração em configuraçãoe desenvolvimento de projetos de CFTV\\
		AXIS COMMUNICATIONS, AXIS, Brasil
	\item[\textbf{2010 - 2010}] Java SE. . (Carga horária: 290h).\\
		Centro de Pesquisa e Qualificação Tecnológica, CPQT, Fortaleza, Brasil
\end{itemize}


\clearpage
\begin{center}	%Linha não tão grossa - média
	\rule{\TlinhaG}{\LlinhaG}
\end{center}

\begin{center}
	\textbf{ATUAÇÂO PROFISSIONAL}\\
\end{center}

\begin{center} %Linha não tão grossa - média
	\rule{\TlinhaM}{\LlinhaM}
\end{center}

\begin{multicols}{2}
	\begin{flushleft}
		\textbf{EMPRESA}\\
		\textit{Buson}\\
	\end{flushleft}
	\vfill
	\begin{flushright}
		\textbf{03/2023 - Atual}\\
	\end{flushright}
\end{multicols}
\begin{flushleft}
	\textbf{Desenvolvedor FullStack Pleno II}\\
	Desenvolvedor FullStack, atuante com HTML, CSS, Javascript, Nodejs, Typescript, VueJs, Svelte (POC), Handlebars,
	Java, Spring Boot. Squad  do e-commerce, operando procedimento de deploy, ajuste de erros de build no jenkins, argos.
\end{flushleft}

\begin{center} %Linha não tão grossa - média
	\rule{\TlinhaM}{\LlinhaM}
\end{center}

\begin{multicols}{2}
	\begin{flushleft}
		\textbf{EMPRESA}\\
		\textit{Buson}\\
	\end{flushleft}
	\vfill
	\begin{flushright}
		\textbf{05/2022 - 02/2023}\\
	\end{flushright}
\end{multicols}
\begin{flushleft}
	\textbf{Desenvolvedor FullStack Pleno}\\
	Desenvolvedor FullStack, atuante com HTML, CSS, Javascript, Nodejs, Typescript, VueJs, Svelte (POC), Handlebars,
	Java, Spring Boot. Squad do PDV e e-commerce
\end{flushleft}

\begin{center} %Linha não tão grossa - média
	\rule{\TlinhaM}{\LlinhaM}
\end{center}

\begin{multicols}{2}
	\begin{flushleft}
		\textbf{EMPRESA}\\
		\textit{Maquiadoro}\\
	\end{flushleft}
	\vfill
	\begin{flushright}
		\textbf{01/2021 - 04/2022}\\
	\end{flushright}
\end{multicols}
\begin{flushleft}
	\textbf{Desenvolvedor FullStack e Diretor}\\
	Desenvolvedor Backend em NodeJs, typescript, atentando-se ao SOLID, utilizando clean architecture, Docker, MongoDB, Postgress, Mysql,
	e Frontend em ReactJs em alguns projetos. Manutenção front-end no e-commerce.
	Sistematização, automação e estudo dos KPIs, atuando como consultor de tecnologia e direcionamento da equipe de TI em prol da estratégia da empresa,
	participando e aplicando a metodologia ágil de projetos com framework SCRUM.
\end{flushleft}

\begin{center} %Linha não tão grossa - média
	\rule{\TlinhaM}{\LlinhaM}
\end{center}

\begin{multicols}{2}
	\begin{flushleft}
		\textbf{EMPRESA}\\
		\textit{Maquiadoro}\\
	\end{flushleft}
	\vfill
	\begin{flushright}
		\textbf{05/2019 - 12/2020}\\
	\end{flushright}
\end{multicols}
\begin{flushleft}
	\textbf{PO/Desenvolvedor e Diretor}\\
	Atuante como PO e desenvolvedor da equipe de TI, participando e aplicando a metodologia ágil de projetos com framework SCRUM,
	desenvolvimento de microssistemas que auxiliam na gestão e correção, bem como automatização de alguns processos da empresa.
	.
\end{flushleft}

\begin{center} %Linha não tão grossa - média
	\rule{\TlinhaM}{\LlinhaM}
\end{center}

\begin{multicols}{2}
	\begin{flushleft}
		\textbf{EMPRESA}\\
		\textit{Maquiadoro}\\
	\end{flushleft}
	\vfill
	\begin{flushright}
		\textbf{08/2018 - 01/2019}\\
	\end{flushright}
\end{multicols}
\begin{flushleft}
	\textbf{Diretor de e-commerce e operações}\\
	Responsável pelas análise de dados e indicadores da empresa bem como estipular as metas com base no planejamento estratégico.
	Responsável pelo mapeamento dos processos e conferência fiscal e tributária. Acompanhamento dos indicadores do negócio, tal como,
	taxa de conversão , tráfego por origem/mídia, ticket médio, taxa de rejeição e os indicadores dos setores.
\end{flushleft}

\begin{center} %Linha não tão grossa - média
	\rule{\TlinhaM}{\LlinhaM}
\end{center}

\begin{multicols}{2}
	\begin{flushleft}
		\textbf{EMPRESA}\\
		\textit{Maquiadoro}\\
	\end{flushleft}
	\vfill
	\begin{flushright}
		\textbf{07/2016 - 07/2018}\\
	\end{flushright}
\end{multicols}
\begin{flushleft}
	\textbf{Diretor Comercial e de Expedição/logística}\\
	Responsável pela expansão logística e pelo setor comercial da empresa. Análise da curva de vendas e estimativa da curva de compra,
	Gestão e supervisão da equipe de compras, bem como análise fiscal e tributária em conjunto com a contabilidade.
	Automatização dos processos e modernização de algumas áreas da empresa.
\end{flushleft}

\begin{center} %Linha não tão grossa - média
	\rule{\TlinhaM}{\LlinhaM}
\end{center}

\begin{multicols}{2}
	\begin{flushleft}
		\textbf{EMPRESA}\\
		\textit{Nexo Engenharia}\\
	\end{flushleft}
	\vfill
	\begin{flushright}
		\textbf{09/2015 - 06/2016}\\
	\end{flushright}
\end{multicols}
\begin{flushleft}
	\textbf{Engenheiro de Mecatrônica}\\
	Responsável pelo setor de Desenvolvimento e Start-Up. Elaboração e execução de projetos e orçamentos de automação predial e industrial, CFTV, Controle de acesso, sistema de detecção e alarme de incêndio, bem como desenvolvimento de programas para controlador lógico programável (CLP) para sistemas de automação predial, industrial e gerenciamento de recursos hídricos, configuração de software de supervisão.\\
\end{flushleft}

\begin{center}%Linha não tão grossa - média
	\rule{\TlinhaM}{\LlinhaM}
\end{center}

\begin{multicols}{2}
	\begin{flushleft}
		\textbf{EMPRESA}\\
		\textit{Nexo Engenharia}\\
	\end{flushleft}
	\vfill
	\begin{flushright}
		\textbf{01/2015 até 09/2015}\\
	\end{flushright}
\end{multicols}
\begin{flushleft}
	\textbf{Analista de Sistemas de Automação}\\
	Desenvolvimento de programas para controlador lógico programável (CLP) para sistemas de automação predial, industrial e gerenciamento de recursos hídricos, 	configuração de software de supervisão, controle de acesso, testes em sensores industriais, configuração de redes industriais. Participação na elaboração e analise projeto de sistemas de automação predial de médio e grande porte, controle de acesso, contagem de pessoas, sistema de detecção e alarme de incêndio, e CFTV, bem como \textit{start-up} dos mesmos.\\
\end{flushleft}

\begin{center}
	\rule{\TlinhaM}{\LlinhaM}
\end{center}

\begin{multicols}{2}
	\begin{flushleft}
		\textbf{EMPRESA}\\
		\textit{Nexo Engenharia}\\
	\end{flushleft}
	\vfill
	\begin{flushright}
		\textbf{06/2014 até 01/2015}\\
	\end{flushright}
\end{multicols}
\begin{flushleft}
	\textbf{Auxiliar de Automação}\\
	Desenvolvimento de programas para controlador lógico programável (CLP) para sistemas de automação predial, industrial e gerenciamento de recursos hídricos, configuração de software de supervisão, controle de acesso, testes em sensores industriais, configuração de redes industriais, \textit{start-up} de sistemas de automação de climatização de grande porte, controle de acesso, contagem de pessoas, sistema de detecção e alarme de incêndio, e CFTV\\
\end{flushleft}

\begin{center}
	\rule{\TlinhaM}{\LlinhaM}
\end{center}

\begin{multicols}{2}
	\begin{flushleft}
		\textbf{EMPRESA}\\
		\textit{Nexo Engenharia}\\
	\end{flushleft}
	\vfill
	\begin{flushright}
		\textbf{12/2012 até 06/2014}\\
	\end{flushright}
\end{multicols}
\begin{flushleft}
	\textbf{Estagiário}\\
	Desenvolvimento de programas para controlador lógico programável (CLP) para sistemas de automação predial, industrial, configuração de software de supervisão, controle de acesso, testes em sensores industriais, configuração de redes industriais, Levantamento de quantitativos e pontos de projetos de automação.\\
\end{flushleft}

\begin{center}
	\rule{\TlinhaM}{\LlinhaM}
\end{center}

\begin{multicols}{2}
	\begin{flushleft}
		\textbf{EMPRESA/INSTITUIÇÃO}\\
		\textit{Laboratório de Processamento de Energia - LPE (IFCE)}\\
	\end{flushleft}
	\vfill
	\begin{flushright}
		\textbf{2012 até 2014}\\
	\end{flushright}
\end{multicols}
\begin{flushleft}
	\textbf{Bolsista}\\
	Desenvolvimento de projetos no âmbito da eletrônica de potência, controle e Energia eólica.\\
\end{flushleft}

\begin{center}
	\rule{\TlinhaM}{\LlinhaM}
\end{center}

\begin{multicols}{2}
	\begin{flushleft}
		\textbf{EMPRESA}\\
		\textit{TS Tecnologia e soluções em automação industrial LTDA}\\
	\end{flushleft}
	\vfill
	\begin{flushright}
		\textbf{11/2011 até 05/2012}\\
	\end{flushright}
\end{multicols}
\begin{flushleft}
	\textbf{Estagiário}\\
	Análise de sistemas de automação industrial, Programação de controladores lógicos programáveis CLP’s, Desenvolvimento de sistema supervisório.\\
\end{flushleft}

\begin{center}
	\rule{\TlinhaM}{\LlinhaM}
\end{center}


\begin{center}
	\textbf{EXPERIÊNCIA PROFISSIONAL COMPLEMENTAR}\\
\end{center}
\begin{itemize}
	\item Desenvolvimento do site usando Nextjs (Reactjs, Eslint, Husky, Staged-lint, stylelint, scss) https://bid.log.br
	\item Desenvolvimento do site wordpress https://psicologarichellysousa.com.br
	\item Ministro de um WORKSHOP  de introdução a microcontroladores e aplicações na FACULDADE FARIAS BRITO (FFB) - 2015
	\item Ministro de um WORKSHOP  de introdução a microcontroladores e aplicações na UNIVERSIDADE DE FORTALEZA (UNIFOR) para os cursos de Engenharia de controle e Automação - 2014
	\item Palestrante com o tema AUTOMAÇÃO  na V AMOSTRA DE PESQUISA EM CIENTIFICA E TECNOLOGIA - 2014.
	\item Ministro de um MINICURSO DE INTRODUÇÃO A AUTOMAÇÃO INDUSTRIAL com duração de 8 horas durante dois dias na SEMANA DA TI do Curso de Tecnologia da Informação na FACULDADE INTERNACIONAL FANOR - 2013
	\item Ministro de um MINICURSO DE INTRODUÇÃO A AUTOMAÇÃO INDUSTRIAL com duração de 12 horas durante 3 dias na SEMANA DA ENGENHARIA ELÉTRICA do Curso de ENGENHARIA ELÉTRICA na FACULDADE INTERNACIONAL FANOR - 2013
\end{itemize}

\begin{center}
	\rule{\TlinhaG}{\LlinhaG}
\end{center}

\vspace{5cm}
\begin{center}
	\noindent Fortaleza, \today  % \noident  indica que esta linha não tem identação  
\end{center}
% \today coloca a data do dia corrente
\end{document}